\documentclass{article}

\usepackage{arxiv}

\usepackage[utf8]{inputenc} % allow utf-8 input
\usepackage[T1]{fontenc}    % use 8-bit T1 fonts
\usepackage{lmodern}        % https://github.com/rstudio/rticles/issues/343
\usepackage{hyperref}       % hyperlinks
\usepackage{url}            % simple URL typesetting
\usepackage{booktabs}       % professional-quality tables
\usepackage{amsfonts}       % blackboard math symbols
\usepackage{nicefrac}       % compact symbols for 1/2, etc.
\usepackage{microtype}      % microtypography
\usepackage{graphicx}

\title{spmodel: Spatial Modeling in \textbf{R} -- Supplementary Material}

\author{
    Michael Dumelle
    \thanks{Corresponding Author}
   \\
    United States \\
    Environmental Protection Agency \\
  200 SW 35th St, Corvallis, OR, 97333 \\
  \texttt{\href{mailto:Dumelle.Michael@epa.gov}{\nolinkurl{Dumelle.Michael@epa.gov}}} \\
   \And
    Matt Higham
   \\
    Department of Math, Computer Science, and Statistics \\
    St.~Lawrence University \\
  23 Romoda Drive, Canton, NY, 13617 \\
  \texttt{\href{mailto:mhigham@stlawu.edu}{\nolinkurl{mhigham@stlawu.edu}}} \\
   \And
    Jay M. Ver Hoef
   \\
    National Oceanic and Atmospheric Administration \\
    Alaska Fisheries Science Center \\
  Marine Mammal Laboratory, Seattle, WA, 98115 \\
  \texttt{\href{mailto:jay.verhoef@noaa.gov}{\nolinkurl{jay.verhoef@noaa.gov}}} \\
  }


% tightlist command for lists without linebreak
\providecommand{\tightlist}{%
  \setlength{\itemsep}{0pt}\setlength{\parskip}{0pt}}



\usepackage{amsmath,amsfonts,amssymb}
\usepackage{bm, bbm}
\begin{document}
\maketitle


\begin{abstract}
Enter the text of your abstract here.
\end{abstract}

\keywords{
    Spatial covariance
   \and
    Linear Model
   \and
    Autoregressive model
  }

\hypertarget{estimation}{%
\section{Estimation}\label{estimation}}

\hypertarget{likelihood-based-estimation}{%
\subsection{Likelihood-based
Estimation}\label{likelihood-based-estimation}}

Minus twice a profiled Gaussian log-likelihood, denoted
\(-2l(\bm{\theta} | \mathbf{y})\) is given by
\begin{equation}\label{eq:ml-lik}
  -2l(\bm{\theta} | \mathbf{y}) = \ln{|\mathbf{\Sigma}|} + (\mathbf{y} - \mathbf{X} \tilde{\bm{\beta}})^\intercal \mathbf{\Sigma}^{-1} (\mathbf{y} - \mathbf{X} \tilde{\bm{\beta}}) + n \ln{2\pi},
\end{equation} where
\(\tilde{\bm{\beta}} = (\mathbf{X}^\intercal \mathbf{\Sigma}^{-1} \mathbf{X})^{-1} \mathbf{X}^\intercal \mathbf{\Sigma}^{-1} \mathbf{y}\).
Minimizing Equation\(~\)\ref{eq:ml-lik} yields
\(\bm{\hat{\theta}}_{ml}\), the maximum likelihood estimates for
\(\bm{\theta}\). Then a closed for solution exists for
\(\bm{\hat{\beta}}_{ml}\), the maximum likelihood estimates for
\(\bm{\beta}\): \(\bm{\hat{\beta}}_{ml} = \tilde{\bm{\beta}}_{ml}\),
where \(\tilde{\bm{\beta}}_{ml}\) is \(\tilde{\bm{\beta}}\) evaluated at
\(\bm{\hat{\theta}}_{ml}\). Unfortunately \(\bm{\hat{\theta}}_{ml}\) can
be badly biased for \(\bm{\theta}\) (especially for small sample sizes),
which impacts the estimation of \(\bm{\beta}\) (Patterson and Thompson
1971). This bias occurs due to the simultaneous estimation of
\(\bm{\beta}\) and \(\bm{\theta}\) To reduce this bias, restricted
maximum likelihood estimation (REML) emerged (Patterson and Thompson
1971; Harville 1977; Wolfinger, Tobias, and Sall 1994). It can be shown
that integrating \(\bm{\beta}\) out of a Gaussian likelihood yields the
restricted Gaussian likelihood used in REML estimation. Minus twice a
restricted Gaussian log-likelihood, denoted
\(-2l_R(\bm{\theta} | \mathbf{y})\) is given by
\begin{equation}\label{eq:reml-lik}
  -2l_R(\bm{\theta} | \mathbf{y}) = -2l(\bm{\theta} | \mathbf{y})  + \ln{|\mathbf{X}^\intercal \mathbf{\Sigma}^{-1} \mathbf{X}|} - p \ln{2\pi} .
\end{equation} Minimizing Equation\(~\)\ref{eq:reml-lik} yields
\(\bm{\hat{\theta}}_{reml}\), the restricted maximum likelihood
estimates for \(\bm{\theta}\). Then a closed for solution exists for
\(\bm{\hat{\beta}}_{reml}\), the restricted maximum likelihood estimates
for \(\bm{\beta}\):
\(\bm{\hat{\beta}}_{reml} = \tilde{\bm{\beta}}_{reml}\), where
\(\tilde{\bm{\beta}}_{reml}\) is \(\tilde{\bm{\beta}}\) evaluated at
\(\bm{\hat{\theta}}_{reml}\).

When all variance parameters are unknown, the overall variance,
\(\sigma^2\), can be profiled out of Equation\(~\)\ref{eq:ml-lik} and
Equation\(~\)\ref{eq:reml-lik}. This reduces the number of parameters
requiring optimization by one, which can dramatically reduce estimation
time. For example, profiling \(\sigma^2\) out of
Equation\(~\)\ref{eq:ml-lik} yields \begin{equation}\label{eq:ml-plik}
  -2l^*(\bm{\theta}^* | \mathbf{y}) = \ln{|\mathbf{\Sigma^*}|} + n\ln[(\mathbf{y} - \mathbf{X} \tilde{\bm{\beta}})^\intercal \mathbf{\Sigma}^{-1} (\mathbf{y} - \mathbf{X} \tilde{\bm{\beta}})] + n + n\ln{2\pi / n}.
\end{equation} After finding \(\hat{\bm{\theta}}^*_{ml}\) a closed form
solution for \(\hat{\sigma}^2_{ml}\) exists:
\(\hat{\sigma}^2_{ml} = [(\mathbf{y} - \mathbf{X} \bm{\tilde{\beta}})^\intercal \mathbf{\Sigma}^{-1} (\mathbf{y} - \mathbf{X} \tilde{\bm{\beta}})] / n\).
Then \(\bm{\hat{\theta}}^*_{ml}\) is combined with
\(\hat{\sigma}^2_{ml}\) to yield \(\bm{\hat{\theta}}_{ml}\) and
subsequently \(\bm{\hat{\beta}}_{ml}\)

Next describe REML adjustments

\hypertarget{semivariogram-based-estimation}{%
\subsection{Semivariogram-based
Estimation}\label{semivariogram-based-estimation}}

\hypertarget{weighted-least-squares}{%
\subsubsection{Weighted Least Squares}\label{weighted-least-squares}}

(Cressie 1985, 1993)

\hypertarget{composite-likelihood}{%
\subsubsection{Composite Likelihood}\label{composite-likelihood}}

(Curriero and Lele 1999)

\hypertarget{hypothesis-testing}{%
\section{Hypothesis Testing}\label{hypothesis-testing}}

\hypertarget{the-general-linear-hypothesis-test}{%
\subsection{The General Linear Hypothesis
Test}\label{the-general-linear-hypothesis-test}}

\hypertarget{contrasts}{%
\subsection{Contrasts}\label{contrasts}}

\hypertarget{random-effects}{%
\section{Random Effects}\label{random-effects}}

\hypertarget{blups}{%
\subsection{BLUPs}\label{blups}}

\hypertarget{references}{%
\section*{References}\label{references}}
\addcontentsline{toc}{section}{References}

\hypertarget{refs}{}
\leavevmode\hypertarget{ref-cressie1985fitting}{}%
Cressie, Noel. 1985. ``Fitting Variogram Models by Weighted Least
Squares.'' \emph{Journal of the International Association for
Mathematical Geology} 17 (5): 563--86.

\leavevmode\hypertarget{ref-cressie1993statistics}{}%
---------. 1993. \emph{Statistics for Spatial Data}. John Wiley \& Sons.

\leavevmode\hypertarget{ref-curriero1999composite}{}%
Curriero, Frank C, and Subhash Lele. 1999. ``A Composite Likelihood
Approach to Semivariogram Estimation.'' \emph{Journal of Agricultural,
Biological, and Environmental Statistics}, 9--28.

\leavevmode\hypertarget{ref-harville1977maximum}{}%
Harville, David A. 1977. ``Maximum Likelihood Approaches to Variance
Component Estimation and to Related Problems.'' \emph{Journal of the
American Statistical Association} 72 (358): 320--38.

\leavevmode\hypertarget{ref-patterson1971recovery}{}%
Patterson, H Desmond, and Robin Thompson. 1971. ``Recovery of
Inter-Block Information When Block Sizes Are Unequal.''
\emph{Biometrika} 58 (3): 545--54.

\leavevmode\hypertarget{ref-wolfinger1994computing}{}%
Wolfinger, Russ, Randy Tobias, and John Sall. 1994. ``Computing Gaussian
Likelihoods and Their Derivatives for General Linear Mixed Models.''
\emph{SIAM Journal on Scientific Computing} 15 (6): 1294--1310.

\bibliographystyle{unsrt}
\bibliography{references.bib}


\end{document}
