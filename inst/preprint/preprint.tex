\documentclass{article}

\usepackage{arxiv}

\usepackage[utf8]{inputenc} % allow utf-8 input
\usepackage[T1]{fontenc}    % use 8-bit T1 fonts
\usepackage{lmodern}        % https://github.com/rstudio/rticles/issues/343
\usepackage{hyperref}       % hyperlinks
\usepackage{url}            % simple URL typesetting
\usepackage{booktabs}       % professional-quality tables
\usepackage{amsfonts}       % blackboard math symbols
\usepackage{nicefrac}       % compact symbols for 1/2, etc.
\usepackage{microtype}      % microtypography
\usepackage{graphicx}

\title{spmodel: Spatial Modeling in \textbf{R}}

\author{
    Michael Dumelle
    \thanks{Corresponding Author}
   \\
    United States \\
    Environmental Protection Agency \\
  200 SW 35th St, Corvallis, OR, 97333 \\
  \texttt{\href{mailto:Dumelle.Michael@epa.gov}{\nolinkurl{Dumelle.Michael@epa.gov}}} \\
   \And
    Matt Higham
   \\
    Department of Math, Computer Science, and Statistics \\
    St.~Lawrence University \\
  23 Romoda Drive, Canton, NY, 13617 \\
  \texttt{\href{mailto:mhigham@stlawu.edu}{\nolinkurl{mhigham@stlawu.edu}}} \\
   \And
    Jay M. Ver Hoef
   \\
    National Oceanic and Atmospheric Administration \\
    Alaska Fisheries Science Center \\
  Marine Mammal Laboratory, Seattle, WA, 98115 \\
  \texttt{\href{mailto:jay.verhoef@noaa.gov}{\nolinkurl{jay.verhoef@noaa.gov}}} \\
  }


% tightlist command for lists without linebreak
\providecommand{\tightlist}{%
  \setlength{\itemsep}{0pt}\setlength{\parskip}{0pt}}



\usepackage{amsmath,amsfonts,amssymb}
\usepackage{bm, bbm}
\begin{document}
\maketitle


\begin{abstract}
Enter the text of your abstract here.
\end{abstract}

\keywords{
    Spatial covariance
   \and
    Linear Model
   \and
    Autoregressive model
  }

\hypertarget{introduction}{%
\section{\texorpdfstring{Introduction
\label{sec:introduction}}{Introduction }}\label{introduction}}

\begin{verbatim}
R> x <- 1
R> x <- list(
+   a = 1
+ )
R> x
\end{verbatim}

\begin{verbatim}
$a
[1] 1
\end{verbatim}

Here we describe the general role of spatial modeling, discuss existing
software, and argue spmodel is a valuable contribution.

\hypertarget{background-and-usage}{%
\section{Background and Usage}\label{background-and-usage}}

\hypertarget{spatial-linear-models}{%
\subsection{Spatial Linear Models}\label{spatial-linear-models}}

\begin{itemize}
\tightlist
\item
  REML citations (Patterson and Thompson 1971; Harville 1977; Wolfinger,
  Tobias, and Sall 1994)
\item
  SV-WLS citations (Cressie 1985, 1993)
\item
  SV-CL citations (Curriero and Lele 1999)
\end{itemize}

\hypertarget{anisotropy}{%
\subsubsection{Anisotropy}\label{anisotropy}}

\hypertarget{spatial-autoregressive-models}{%
\subsection{Spatial Autoregressive
Models}\label{spatial-autoregressive-models}}

\hypertarget{prediction}{%
\subsection{Prediction}\label{prediction}}

\hypertarget{neighborhood-indexing-for-big-data}{%
\subsection{Neighborhood Indexing for Big
Data}\label{neighborhood-indexing-for-big-data}}

\hypertarget{the-local-list}{%
\subsubsection{The local list}\label{the-local-list}}

\hypertarget{random-effects}{%
\subsection{Random Effects}\label{random-effects}}

\hypertarget{partition-factors}{%
\subsection{Partition Factors}\label{partition-factors}}

\hypertarget{initial-values-and-known-values}{%
\subsection{Initial Values and Known
Values}\label{initial-values-and-known-values}}

\hypertarget{simulating-gaussian-random-variables}{%
\subsection{Simulating Gaussian Random
Variables}\label{simulating-gaussian-random-variables}}

\hypertarget{discussion}{%
\section{Discussion}\label{discussion}}

\hypertarget{data-and-code-availability}{%
\section*{Data and Code Availability}\label{data-and-code-availability}}
\addcontentsline{toc}{section}{Data and Code Availability}

\hypertarget{acknowledgements}{%
\section*{Acknowledgements}\label{acknowledgements}}
\addcontentsline{toc}{section}{Acknowledgements}

\hypertarget{references}{%
\section*{References}\label{references}}
\addcontentsline{toc}{section}{References}

\hypertarget{refs}{}
\leavevmode\hypertarget{ref-cressie1985fitting}{}%
Cressie, Noel. 1985. ``Fitting Variogram Models by Weighted Least
Squares.'' \emph{Journal of the International Association for
Mathematical Geology} 17 (5): 563--86.

\leavevmode\hypertarget{ref-cressie1993statistics}{}%
---------. 1993. \emph{Statistics for Spatial Data}. John Wiley \& Sons.

\leavevmode\hypertarget{ref-curriero1999composite}{}%
Curriero, Frank C, and Subhash Lele. 1999. ``A Composite Likelihood
Approach to Semivariogram Estimation.'' \emph{Journal of Agricultural,
Biological, and Environmental Statistics}, 9--28.

\leavevmode\hypertarget{ref-harville1977maximum}{}%
Harville, David A. 1977. ``Maximum Likelihood Approaches to Variance
Component Estimation and to Related Problems.'' \emph{Journal of the
American Statistical Association} 72 (358): 320--38.

\leavevmode\hypertarget{ref-patterson1971recovery}{}%
Patterson, H Desmond, and Robin Thompson. 1971. ``Recovery of
Inter-Block Information When Block Sizes Are Unequal.''
\emph{Biometrika} 58 (3): 545--54.

\leavevmode\hypertarget{ref-wolfinger1994computing}{}%
Wolfinger, Russ, Randy Tobias, and John Sall. 1994. ``Computing Gaussian
Likelihoods and Their Derivatives for General Linear Mixed Models.''
\emph{SIAM Journal on Scientific Computing} 15 (6): 1294--1310.

\bibliographystyle{unsrt}
\bibliography{references.bib}


\end{document}
